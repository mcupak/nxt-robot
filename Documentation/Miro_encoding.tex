\subsection{Code representation}
How do you encode instructions? What are adventeges and disadvanteges of this representations? How is it read? Do you use some special symbols?

I see two ways how to make coding (I already nicknamed the language \emph{GreyDot}, do you agree?):
\begin{itemize}
\item Use scale for gray with 10 steps.
\item Use triples of black (0) and gray (1).
\end{itemize}
First option would be better, but we need to test if we can always calibrate the sensor so it reads correct instructions no matter what light conditions are - but it should work, when I tested it, it was pretty sharp. In this case we have 9 steps for different instructions and 1 for background. \\
Second option uses only three steps - black for 0, grey for 1 and white for background and I am pretty positive that this option would work even without self calibration. We would than have 8 combinations.

\subsection{Special symbols}
There is at leas one symbol left for each line - these will be used as special symbols - for now we may need these symbols:
\begin{enumerate}
\item maybe - Start color calibration (should be black-black).
\item maybe - Start path calibration.
\item sure - Start reading instructions.
\item sure - End reading instructions and execute.
\end{enumerate}