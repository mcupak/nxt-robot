\section{Instructions execution}
The final phase of the robot's program aims to interpret the newly decoded
binary data. This works as expected: the instruction interpreter sets its
Program Counter (PC) to the first data \textit{byte} (i.e. 2-bit value)
which determines what should be executed and then repeats the whole
process until it moves out of the array of instructions. Every
instruction moves the PC one \textit{byte} further, instructions have
access to the PC and can (theoretically) alter it to perform loops.

\subsection{Instruction set}
Although it would be possible to prepare a "Turing-complete" set
of instructions, our goal was to showcase that the robot does what it is
told using only a few bytes.
The same goes for the number of instructions - we used 2-\textit{byte}
instructions (first its function, then its parameter), but it is
possible to have longer / shorter / variable length instructions.

\begin{center}
\begin{tabular}{|c|c||l|}
\hline
\textbf{Byte 1}   &   \textbf{Byte 2}   &   \textbf{Description}\\
\hline
\hline
    0   &   0       &   Go forward\\
        &   1       &   Go forward left\\
        &   2       &   Go forward right\\
        &   3       &   Repeat LAI 2x\\
\hline
    1   &   0       &   Go back\\
        &   1       &   Go back left\\
        &   2       &   Go back right\\
        &   3       &   Repeat LAI 5x\\
\hline
    2   &   X       &   Play tone (X+1) * 100~Hz\\
\hline
    3   &   X       &   Delay time (X+1) * 1000~ms\\
 \hline
 \end{tabular}
\end{center}

Every standard instruction is executed for 1000~ms. Sound playing
instructions are executed in parallel and therefore take 0~ms to execute.

The \textit{Repeat LAI Xx} performs the \textit{Last Action Instruction}
$X$-times. This servers as a simple way to perform movement instructions
multiple times using fewer data \textit{bytes}.

Because every possible 2-\textit{byte} value has a meaning assigned,
it is not possible for the user program to cause an error along the
execution based on a wrong \textit{byte} value. However if the number
of \textit{bytes} is not even, error occurs on the very last instruction.


