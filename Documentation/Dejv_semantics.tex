\subsection{Semantics}
What are the semantics of the code? What are their capabilities?

I would use this two sets of instructions:
\begin{itemize}
\item Movement
\begin{enumerate}
\item Move forward straight.
\item Move forward left to make 30� turn.
\item Move forward right to make 30� turn.
\item Move backward straight.
\item Move backward left to make 30� turn.
\item Move backward right to make 30� turn.
\item Stay / Stop.
\end{enumerate}
\item Sound
\begin{enumerate}
\item Beep note A for 1/3 of period.
\item Beep note A for 2/3 of period.
\item Beep note A for 3/3 of period.
\item Beep note B for 1/3 of period.
\item Beep note B for 2/3 of period.
\item Beep note B for 3/3 of period.
\item Be silent.
\end{enumerate}
\end{itemize}
One line of instructions would be for movement only, the other one would be for sound only. \\
After reading instructions our robot would always execute both instructions at once for a fixed period of time. \\

\subsubsection{Example}
Using only numbers of instructions following code:
\begin{verbatim}
222222
363636
\end{verbatim}
would make a circling firetruck :)